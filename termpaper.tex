% This file is based on the "sig-alternate.tex" V1.9 April 2009
% This file should be compiled with V2.4 of "sig-alternate.cls" April 2009

\documentclass{sig-alternate}

\usepackage{url}
\usepackage{color}
\usepackage{enumerate}
\usepackage{balance}
\permission{}
\CopyrightYear{2013}
%\crdata{0-00000-00-0/00/00}
\begin{document}

\title{The Effects of Blah Property on Blah}
\numberofauthors{1}
\author{
\alignauthor
Anonymous
}
\date{25 August 2013}
\maketitle
\begin{abstract}
  The abstract should be one or two paragraphs that summarize your
  paper. Abstracts are read independently from the rest of the paper
  so you cannot cite any other paper here. Study other abstracts in
  the papers you are reading to understand what an abstract should
  really means.
\end{abstract}

\section{Overview}
\label{overview}

Use this section to describe the session topic area and
present an overview of the papers.

(For Phase 1, write down the needed
paragraph in this section. For the other sections, leave the section
headers in place but delete the text within the sections. Of course,
you need to make use of BibTeX to generate the References section for
your selected papers.)

Provide the roadmap for the remaining sections of the
paper. For example, you can state that Section \ref{common
  themes} presents common themes in the papers being
surveyed and section \ref{discordant themes} presents
themes where the papers disagree.  

\section{Common Themes}
\label{common themes}

Use this section to describe what is common about the
problem or solution approaches in the selected papers.

\section{Discordant Themes}
\label{discordant themes}

Use this section to describe the issues that the papers
disagree about.

\section{Non-overlapping Themes}
\label{non-overlapping themes}

Use this section to describe issues that the papers deal
with that are neither common nor disagreements.

\section{Final Remarks}
\label{final remarks}

Use this section to summarize your conclusions.

Describe what {\bf you} think about these selected papers
based on your understanding. You should include what you
liked and did not like about the technical content, style of
writing, and so on.

\subsection{Tables, Figures, and Citations/References}

Tables, figures, and citations/references in technical
documents need to be presented correctly. As many students
are not familiar with using these objects, here is a quick
guide extracted from the ACM style guide.

\begin{table}
\centering
\caption{Feelings about the Paper}
\label{FEELINGS}
\begin{tabular}{|l|r|l|} \hline
Flavor&Percentage&Comments\\ \hline
Paper 1 &  10\% & Loved it a lot\\ \hline
Paper 1 &  20\% & Disliked it immensely\\ \hline
Paper 1 &  30\% & Didn't care one bit\\ \hline
Paper 1 &  40\% & Duh?\\ \hline
\end{tabular}
\end{table}


First, note that figures in the term paper must be original,
that is, created by the student: please do not cut-and-paste
figures from the papers you are reading. Second, if you do
need to include figures, they should be handled as
demonstrated here. State that Figure \ref{sample graphic} is
a simple illustration used in the ACM Style sample
document. Figures are never below or above. Incidentally,
table captions are above the table and figure captions are
below the figure. For example, you can refer to a table as Table
\ref{FEELINGS}, which presents people's feelings.

\begin{figure}[htb]
\label{sample graphic}
\begin{center}
\includegraphics[width=2in]{fly.jpg}
\caption{A sample black \& white graphic (JPG).}
\label{sample graphic}
\end{center}
\end{figure}

Finally, citing documents needs to be done properly too. For
example, a paper by Mic Bowman, Saumya K. Debray, and Larry
L. Peterson could be cited as Bowman, Debray, and Peterson
\cite{bowman:reasoning}. A set of papers could collectively
be cited as the literature in this area consists of several
interesting papers
\cite{braams:babel,clark:pct,herlihy:methodology}.

You will find the BibTeX entries needed for the research papers you are
reading online, or you can write your own versions easily and add them
to the $termpaper.bib$ file in the folder.

The list of all references will be generated in ACMRef
standard style using the \LaTeX{}/BibTeX. Note that you
need to first the following sequence to get the paper
compiled correctly:

\begin{enumerate}
\item {\tt latex} {\em termpaper}
\item {\tt bibtex} {\em termpaper}
\item {\tt latex} {\em termpaper}
\item {\tt latex} {\em termpaper}
\end{enumerate}

Finally, one of the easier to install Windows \LaTeX{} installations
is proTeXt (\url{http://www.tug.org/protext/}), and a MacOS X
installation is TeXshop
(\url{http://pages.uoregon.edu/koch/texshop/}). You could also try to
find a distribution at \LaTeX{} project website
(\url{http://latex-project.org/ftp.html}). Of course, the correct way
to cite a reference with a URL is to cite it like this \LaTeX{}
project \cite{Latex-distro}.

\bibliographystyle{abbrv}
\bibliography{termpaper}
% You must have a proper ".bib" file
%  and remember to run:
% latex bibtex latex latex
% to resolve all references
\balance
\end{document}
