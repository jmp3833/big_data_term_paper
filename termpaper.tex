% This file is based on the "sig-alternate.tex" V1.9 April 2009
% This file should be compiled with V2.4 of "sig-alternate.cls" April 2009

\documentclass{sig-alternate}

\usepackage{url}
\usepackage{color}
\usepackage{enumerate}
\usepackage{balance}
\permission{}
\CopyrightYear{2015}
%\crdata{0-00000-00-0/00/00}
\begin{document}

\title{The Impact and Discoveries of Data Mining in the Field of Social Media}
\numberofauthors{1}
\author{
\alignauthor
Justin M. Peterson
}
\date{15 June 2015}
\maketitle
\begin{abstract}
\end{abstract}

\section{Overview}
\label{overview}

As more and more companies are leveraging use of big data within
enterprise software applications, social media sites have been high in the 
ranks of contributing groups of data mining and analytics techniques. Some of the
greatest advantages that the big data movement has provided have been through
unanticipated coorelations between data sets that seemed to have no relation
to one another. The social media space has provided troves of publicly accessible 
data that can generate links between seeminly uncommon metrics such as citation prediction
of articles and a user's Twitter feed. \cite{PMC:articles/PMC3278109}

The WSDM conference held in Shanghai China \cite{DBLP:conf/wsdm/2015}
covered many facets of social media and uncovered some uncommon 
relationships between collections of data. While the sheer volume
of data that is publicly available on these networks is cause for analysis, many enterprise
companies are still having difficulty generating new insights from these data sets that
will have a direct, positive impact on their organization.  

Provide the roadmap for the remaining sections of the
paper. For example, you can state that Section \ref{common
  themes} presents common themes in the papers being
surveyed and section \ref{discordant themes} presents
themes where the papers disagree.  

\section{Common Themes}
\label{common themes}

\section{Discordant Themes}
\label{discordant themes}

\section{Non-overlapping Themes}
\label{non-overlapping themes}

\section{Final Remarks}
\label{final remarks}

\subsection{Tables, Figures, and Citations/References}

\bibliographystyle{abbrv}
\bibliography{termpaper}
% You must have a proper ".bib" file
%  and remember to run:
% latex bibtex latex latex
% to resolve all references
\balance
\end{document}
