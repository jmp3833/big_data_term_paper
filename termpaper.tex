% This file is based on the "sig-alternate.tex" V1.9 April 2009
% This file should be compiled with V2.4 of "sig-alternate.cls" April 2009

\documentclass{sig-alternate}

\usepackage{url}
\usepackage{color}
\usepackage{enumerate}
\usepackage{balance}
\permission{}

\CopyrightYear{2013}
%\crdata{0-00000-00-0/00/00}
\begin{document}
\title{The Impact and Discoveries of Data Mining in the Field of
Social Media}
\numberofauthors{1}
\author{
\alignauthor
Justin M. Peterson
}
\date{25 July 2015}
\maketitle
\begin{abstract}
As more and more companies are leveraging the use of big data within
enterprise software applications, social media sites have been high in the 
ranks of contributing groups of data mining and analytics techniques. Some of the
greatest advantages that the big data movement has provided have been through
unanticipated correlations between data sets that seemed to have no relation
to one another. The social media space has provided troves of publicly accessible 
data that can generate links between seemingly uncommon metrics such as citation prediction
of articles and a user's Twitter feed. 

The WSDM conference held in Shanghai China
covered many facets of social media and uncovered some uncommon 
relationships between collections of data. While the sheer volume
of data that is publicly available on these networks is cause for analysis, many enterprise
companies are still having difficulty generating new insights from these data sets that
will have a direct, positive impact on their organization. In the social networks session
of the WSDM conference many researchers have presented strategies of data mining and analysis 
that can help to improve the state of any product that uses social media to drive key business 
decisions. 

This paper describes the strategies and common themes between pieces of research that were presented
in the social media section of the WSDM conference. In addition to evaluating these themes, content from additional
resources is included in order to validate the claims made in each paper analyzed. While the analysis 
performed by these researchers is solid and accredited, questions and challenges that have arisen from 
evaluating their research are also included in this paper.
\end{abstract}

\section{Common Themes}
\label{common themes}
While the six papers presented in the social media section of WSDM \cite{DBLP:conf/wsdm/2015} may be working to achieve progress in
different sectors of social data mining and analysis, each paper works to reap additional 
benefits from already existing social media data. 

From the perspective of product advertisement, marketing teams are chomping at the bit for real time
information and insightful pieces of data to make key decisions. Several marketing teams already use
social media data such as public profile info, and links between profiles in the form of friendships and 
followers as cause for analysis. Through new research published at WSDM, it has been shown that there may be just as much
to learn from the lack of a link between two profiles as there is to earn from established relationships
in social media. 

In work published by S. Lattanzi and Y. Singer\cite{DBLP:conf/wsdm/LattanziS15} the authors work to confirm the
existence of the friendship paradox, where 'most' people have fewer friends than they do, on average. If this
were to be proven true in the context of a social media graph, its existence could lead to the ability
to tap into more heavy hitting profiles for product advertising and information distribution.
There is an additional overview of some of the issues found during this analysis technique in the Non-overlapping Themes section of this paper. 

Unfortunately while the friendship paradox seems to hold in test samples of social media graphs, this may
only be because of the fact that the range of user's average number of relationships on social media
produces a heavily tailed distribution. While the paper does not go in depth on why this distribution is heavily
tailed, there are many obviously apparent factors. For example, it is common that many social media 
users will follow their friends, as well as accounts that represent celebrities, large product brands, 
or companies of interest. Accounts with a more broad range of followers can tend to dominate results 
and be identified as 'key targets' to distribute information through social media. Another interesting
perspective provided by this research suggests that the number of relationships associated with a user profile
may not be the only effective way to decide how to disburse information through a group for maximum influence.
This concept is covered in greater depth in a cited paper by D. Kempe, J. Kleinberg, and E. Tardos of Cornell university.
\cite{kdd03-cornell}

In addition, research performed by J. Liu, C. Aggarwal, and J. Han \cite{Liu:2015:INC:2684822.2685323} disregards individual relationships between profiles
and instead works to discover sets of communities within a social network with limited access to resources. These
discovered groups can be used to for classification of a set of social media users, and can help to discover 
target audiences for information distribution. While the premise of the paper is to incorporate the act of
community detection with the process of discovering entire chunks of a social network graph, the results of the 
research go hand in hand with improving the ability to reach consumers. 

The works also hold the common theme of predicting behavior of a user based on both their personal social
media activity and their interactions with other user's on the site. There are some data points that may not
be publicly available on all social media sites, but can be inferred to improve the understanding of a profile.

Two examples of this principle include the ability to discover negative links or associations between user profiles 
\cite{Tang:2015:NLP:2684822.2685295}
and the use of sentiment analysis and past network use to detect the use of sarcasm in posts. A negative link
can include concepts such as a down vote, a statement saying a resource is untrusted, or a blocked user
depending on which social media graph is being described. While negative links and sarcasm are not the first
strategies that come to mind when attempting to promote an idea or product, they can be used in order to 
avoid mistakes and to provide more curated content to consumers. 

Take a major flight provider for example. While the airline may work towards a comfortable and safe experience
for every passenger, the passenger may have complaints about their trip. It may be common based on a 
user's past history that they will convey this frustration through sarcasm which could be picked up by automated
social media accounts and determined to be an positive comment through sentiment analysis. This can cause
many PR issues within an organization. Using negative link prediction to discover a hostile relationship between
a customer and this airline could also help to reduce such issues. Both of these strategies are covered in greater depth
in the Non-overlapping Themes section of the paper.

\section{Discordant Themes}
\label{discordant themes}
Each paper presented at the WSDM conference under the social media section tackled a unique challenge. 
Progress was made in many different directions,
improving researcher's abilities to mine and analyze social media data for valuable results. While there
are no points in each paper that directly contradict one another, there are certain points where
researcher's analysis strategies diverge.

The main point where the papers move in different directions is in how they each value certain data
points when making their analysis. Researchers have decided to either use only user profile information, 
relationships between profiles, groups of profiles, or a combination of each to perform their analysis. 
Some pieces of research such as the breakdown of sarcasm analysis on Twitter profiles combined a mix
of sentiment analysis as well as a calculated index of how likely the user is to perform sarcasm. This content-centric
strategy shares traits with analysis performed to detect negative links between profiles. Other research including community
discovery and influence of random neighbors rely on data associated with a user profile rather 
than the content that the user is producing.

It is interesting to note that both content-centric approaches to social media data analysis that
were covered in the conference produced self-named frameworks to help outline desired outcomes
and analysis strategies. The NElP (Negative Link Prediction) framework takes in a collection
of formed opinions calculated over several user interactions and profile posts. This information is
then used to determine where negative associations between profiles may exist. In addition, a heavy 
association is placed between negative comments produced by a user and a negative link between that 
user and another profile that is the target of a negative comment. 

The SCUBA (Sarcasm Classification Using Behavioral Modeling Approach) defined to detect
the likelihood of a specific tweet being sarcastic also combines the content generated
from multiple user posts. This analysis takes several factors into account, 
including the number of swear words, appearance of emojis, and past tweets that have been
determined to be of negative sentiment. The paper breaks down the analysis
of sarcasm into multiple subgroups and expands on existing analysis techniques, opposed to using only lexical parsing to 
improve accuracy when compared to other methods.

Another split between the research segments was that some of the works attempted to provide
results that were generic enough that they could be abstracted out to many community based graph
problems, rather than just social media in general. The prime example of this comes from the work regarding
network and community detection. Discovering a community of nodes with dense connections to one another and
a small number of connected routes to other communities has proven to be useful when hunting for more information in
structures such as DNA sequences and terrorist networks. The research can be described both as a graph clustering
algorithm than a social media experiment. \cite{GraphClustering} 

\section{Non-overlapping Themes}
\label{non-overlapping themes}
There are several concepts expressed in each topic covered in this WSDM section. This section aims
to identify highlights that make each piece of research unique.

\subsection{Community Discovery}
\label{community discovery}

The community detection research presented values the importance of discovery of communities within a graph as highly
as the actual discovery of each node in the graph. While it seems that discovering a community structure would only be
useful if the entirety of the graph has been discovered, sometimes this is not the case. 

As an example, view this problem through the lens of Facebook. As stated in the community discovery research, the entire Facebook
graph is not viewable at any given time. A profile can be marked as private, blocking off access to many pieces of 
information such as the relationships between that profile and other nodes that have been marked as friends. This 
can make it particularly tricky to discover more than a large chunk of the graph at any given time. Pair this with the sheer
size of the Facebook user graph, and it would be near impossible to have an entire graph snapshot at any given time!

Even with this principle, community discovery can tell a researcher many properties of a given graph. If a set of communities are discovered on a small subsection of a massive graph such as Facebook, the findings can be abstracted out to analyze other groups of
nodes that have not yet been discovered. This can be useful in generalizing the interests of users who fall under the same
community, and can also allow researchers to identify other communities with the same characteristics. Fields such as marketing 
and security could benefit from these types of analysis.

\subsection{Sarcasm Detection and Grouping}
\label{sarcasm detection and grouping}

\subsection{Negative Links and Positive Link Prediction}
\label{negative links and positive link prediction}

\subsection{Sample Set Choice for Random Social Network Neighbors}
\label{sample set choice for random social network neighbors}

\section{Final Remarks}
\label{final remarks}
Many key points have been discussed and evaluated in the social media section of the WSDM conference. 
Some of these can even be used to make observations on more generic graph problems rather
than just viewing the solutions from a purely social perspective. There is a vast amount of data
that is already available about users that can be used to make business decisions, promote products,
or generate and distribute ideas. The most common information that is available (public profile info, friendships, etc...)
may not be the only pieces of information that can be used to make decisions. There is forward progress
made by making inferences on freely available data, or by using public sample subsets that are
available at a lower computational cost. Techniques such as negative link and sarcasm detection take a great deal
of analysis on past user interactions and posted content, but they avoid costly PR disasters and false
identifications in several fields.

In addition, some techniques are disregarding content centric pieces of data all together and evaluating
the relationships between social media graph nodes. These edges (or lack thereof) can be used to reveal
common communities of people in pieces of a network that are not completely public with a small sample
set. Random neighbors to a node in a social media graph can be identified as beneficial to the distribution
of information, even if the neighbor is not connected to the user within the graph. 

While there is a great deal of progress, it will still be some time until these strategies can be 
thoroughly tested and implemented in commercial products. Each strategy has been tested and 
verified on subsets of social network data or home grown data sets as a proof of concept, but
the real world performance of these techniques is yet to be determined. The WSDM conference and
all of its speakers and researchers should be proud of the achievements they have made as they will
be used for future research and practical application. 

\bibliographystyle{abbrv}
\bibliography{termpaper}
% You must have a proper ".bib" file
%  and remember to run:
% latex bibtex latex latex
% to resolve all references
\balance
\end{document}
